\documentclass[12pt,a4paper,oneside,brazil]{abntex2}
\usepackage[brazil]{babel}
\usepackage[utf8]{inputenc}
\usepackage[osf]{mathpazo}
\renewcommand{\familydefault}{\rmdefault}
\pagestyle{headings}
\setcounter{secnumdepth}{3}
\usepackage{lmodern}
\setcounter{tocdepth}{3}
\usepackage{amsmath, amsfonts, amssymb, amsthm, array}
\usepackage{float} 
\usepackage{calc, cases}
\usepackage[makeroom]{cancel}
\usepackage{nicefrac}

\usepackage{booktabs} 
\usepackage{subfig}
\usepackage{latexsym}
\usepackage[active]{srcltx}
\usepackage{graphicx} 
\graphicspath{ {figures/} }
\usepackage{indentfirst}
\usepackage{xcolor} 
\usepackage{url} 
\usepackage{relsize} 
\usepackage{microtype}
\OnehalfSpacing 
\makeatletter 
\pdfpageheight\paperheight 
\pdfpagewidth\paperwidth 
\providecommand{\tabularnewline}{\\} 
\raggedbottom 
\bookmarksetup{numbered} 
\setlength{\parindent}{1.3cm} 
\renewcommand{\cftsubsectionfont}{\footnotesize\normalfont\rmfamily} 
\renewcommand{\cftsectionfont}{\normalfont\rmfamily} 
\renewcommand{\ABNTEXchapterfont}{\rmfamily\fontseries{b}\selectfont} 
\usepackage{braket}
\usepackage{tikz}
\usepackage{tikz-3dplot}
\usetikzlibrary{hobby}
\usetikzlibrary{angles}
\usetikzlibrary{quotes}
\usepackage{eufrak}
\usepackage{epstopdf}
\usepackage[makeroom]{cancel}
\usepackage{physics}

%% Índices

\theoremstyle{definition}
\newtheorem{defin}{Definição}
\numberwithin{defin}{section}

\newtheorem{thm}{Teorema}
\numberwithin{thm}{section}

\newtheorem{notation}{Notação}
\numberwithin{notation}{section}

\theoremstyle{remark}

\newtheorem*{sol}{Solução}

\newtheorem{exmp}{Exemplo}
\numberwithin{exmp}{section}

\newtheorem*{obs}{Observação}
\newtheorem*{obss}{Observações}

\newtheorem*{prop}{Propriedade}
\newtheorem*{props}{Propriedade}

\newtheorem{p}{Proposição}
\numberwithin{p}{section}

\newtheorem{lema}{Lema}
\numberwithin{lema}{section}


%% Exemplos Especiais

\newtheorem*{exmp1}{Exemplo (Comutador)}


\newcommand\restr[2]{{
		\left.\kern-\nulldelimiterspace 
		#1 
		\vphantom{\big|} 
		\right|_{#2} 
}}

\newcommand{\xRightarrow}[2][]{\ext@arrow 0359\Rightarrowfill@{#1}{#2}}

\newcommand\numeq[1]%
{\stackrel{\scriptscriptstyle(\mkern-1.5mu#1\mkern-1.5mu)}{=}}

% Esse pacote pode ser editado. Por exemplo, onde está escrito {Autor}, dentro dos colchetes você pode colocar o seu nome, e assim por diante.

\hypersetup{
	pdftitle={Relatório de Iniciação Científica}, 
	pdfauthor={ARTHUR DE SOUZA MOLINA},
	pdfsubject={\imprimirpreambulo},
	pdfcreator={abnTeX2},
	pdfkeywords={Ferramentas para Explorar
		a Radiação Cósmica de Fundo e suas Anomalias}{Cosmologia}{Cálculo Numérico}{Estatística}{Correlação Cruzada},
	colorlinks=true,
	linkcolor=black, %Você pode mudar os cores dos links, i.e red, green, blue, etc.
	citecolor=black,
	filecolor=black,
	urlcolor=black,
	bookmarksdepth=4
}

\makeatother 



%% Novo estilo
\makepagestyle{estilo_pretextual} %%% escolha um nome
\makeevenhead{estilo_pretextual}{}{}{\ABNTEXfontereduzida \textbf \thepage}
\makeoddhead{estilo_pretextual}{}{}{\ABNTEXfontereduzida \textbf \thepage}

%% Customiza comando \pretextual
\renewcommand{\pretextual}{
	\pagenumbering{roman} %%% ou \pagenumbering{Roman}
	\aliaspagestyle{chapter}{estilo_pretextual}
	\pagestyle{estilo_pretextual}
	\aliaspagestyle{cleared}{empty}
	\aliaspagestyle{part}{estilo_pretextual}
}

\begin{document}

	\bookmarksetupnext{rellevel=-1}
	
	\pdfbookmark[1]{Capa}{Capa}
	\thispagestyle{empty}
	\begin{center}
		\includegraphics[width=14.8cm]{figuras/logo} 
		\par\end{center}
	
	\begin{center}
		{\color{green!45!black} \rule{1\columnwidth}{1.5mm}}
		\par\end{center}
	
	\begin{center}
		\medskip{}
		\par\end{center}
	
	\begin{center}
		{\Large{}ARTHUR DE SOUZA MOLINA}
		\par\end{center}{\Large \par}
	
	\begin{center}
		\vfill{}
		\par\end{center}
	
	\begin{DoubleSpace}
		\begin{center}
			\textbf{\Large{}Ferramentas para Explorar
				a Radiação Cósmica de Fundo e suas Anomalias}
			\par\end{center}{\Large \par}
	\end{DoubleSpace}
	
	\begin{center}
		\vfill{}
		\par\end{center}
	
	\begin{center}
		{\color{green!45!black} \rule{1\columnwidth}{1.5mm}}
		\par\end{center}
	
	\begin{center}
		Londrina \\
		2021
		\par\end{center}
	
	\cleardoublepage{}
	
	\pdfbookmark[1]{Folha de rosto}{Folha de rosto}
	\thispagestyle{empty}
	\begin{center}
		{\Large{}ARTHUR DE SOUZA MOLINA}
		\par\end{center}{\Large \par}
	
	\begin{center}
		\vfill{}
		\par\end{center}
	
	\begin{DoubleSpace}
		\begin{center}
			\textbf{\Large{}Ferramentas para Explorar a Radiação Cósmica de Fundo e suas Anomalias}
			\par\end{center}{\Large \par}
	\end{DoubleSpace}
	
	\begin{center}
		\vfill{}
		
		\par\end{center}
	
	\noindent \begin{flushright}
		\begin{minipage}[c]{9.5cm}%
			Relatório de Iniciação Científica apresentado ao Departamento de Física da Universidade Estadual de Londrina, como requisito para o Programa de Iniciação Científica.
			
			Orientador: Prof. Dr. Sandro Dias Pinto Vitenti
		\end{minipage}
		\par\end{flushright}
	
	\vfill{}
	
	\begin{center}
		Londrina\\
		2021
		\par\end{center}
	
	\newpage{}
	
	\begin{folhadeaprovacao}
		\thispagestyle{empty}
		\begin{center}
			{\large{}ARTHUR DE SOUZA MOLINA}
			\par\end{center}{\large \par}
		
		\begin{center}
			\vfill{}
			\par\end{center}
		
		\begin{DoubleSpace}
			\begin{center}
				\textbf{\large{}Ferramentas para Explorar a Radiação Cósmica de Fundo e suas Anomalias}
				\par\end{center}{\large \par}
		\end{DoubleSpace}
		
		\begin{center}
			\vfill{}
			
			\par\end{center}
		
		\noindent \begin{flushright}
			\begin{minipage}[c]{9.5cm}%
				Relatório de Iniciação Científica apresentado ao Departamento de Física da Universidade Estadual de Londrina, como requisito para o Programa de Iniciação Científica.
				\begin{center}
					\vspace{1cm}
					\textbf{BANCA EXAMINADORA}
					\par\end{center}
				\vspace{1cm}
				
				\begin{SingleSpace}
					\begin{center}
						\underline{\hspace{9cm}}\\
						Orientador: Prof. Dr. Sandro Dias Pinto Vitenti \\
						Universidade Estadual de Londrina - UEL
						\par\end{center}
					\begin{center}
						\vspace{1cm}
						\underline{\hspace{9cm}}\\
						Prof. Dr. ...\\
						Universidade Estadual de Londrina - UEL
						\par\end{center}
					\begin{center}
						\vspace{1cm}
						\underline{\hspace{9cm}} \\
						Prof. Dr. ... \\
						Universidade Estadual de Londrina - UEL
						\par\end{center}
				\end{SingleSpace}
				\vspace{1cm}
				
				\begin{center}
					Londrina, \today %\_\_\_\_\_ de \_\_\_\_\_\_\_\_\_\_\_\_\_\_\_\_ de \_\_\_\_\_\_\_. 
					\par\end{center}%
			\end{minipage}
			\par\end{flushright}
		
	\end{folhadeaprovacao}
	\begin{agradecimentos}
		Agradeço a minha querida mãe pelo apoio incondicional, carinho, amor, estar comigo sempre e ser uma das pessoas mais importantes da minha vida.
		
		Agradeço ao meu orientador, Prof. Dr. Sandro Dias Pinto Vitenti por me conceder a oportunidade de produzir um projeto de iniciação cientifica, mesmo neste período de dificuldades que enfrentamos diariamente em virtude da pandemia de COVID-19.
		
		  
	\end{agradecimentos}
		
	\begin{SingleSpace}
		\noindent MOLINA, Arthur de Souza \textbf{Ferramentas para Explorar a Radiação Cósmica de Fundo e suas Anomalias}.
		2021. Relatório de Iniciação Científica para o Programa de Iniciação Científica \textendash{}
		Universidade Estadual de Londrina, Londrina, 2021.
	\end{SingleSpace}
	
	\setlength{\absparsep}{18pt}
	
	\begin{resumo}
		
		Com surgimento de diversos modelos cosmológicos, modificações no modelo padrão para explicar fenômenos ou novas teorias físicas e a evolução da tecnologia, tanto na questão da acessibilidade quanto na capacidade de processamento em computadores, surge a demanda por códigos de cosmologia que permitem fazer simulações, a previsão de resultados, verificar a coerência estatística em relação aos dados observacionais, cálculos mais otimizados para os conjuntos de dados que crescem exponencialmente com a produção de experimentos e o consenso dos resultados obtidos em cálculos numéricos entre os grupos de pesquisa. Neste trabalho, estudamos a descrição da formação de estruturas em grandes escala no universo com um tratamento Newtoniano através da instabilidade gravitacional e analisamos os códigos relacionados à correlações cruzadas das bibliotecas de cálculo numérico CCL e NumCosmo, afim de produzir um \textit{jupyter notebook} utilizando um algoritmo de validação do CCL como base para criar um teste de comparação dos cálculos numéricos feitos pelas bibliotecas utilizando o mesmo conjunto de dados, documentar a precisão entre as bibliotecas e os métodos com gráficos e verificar a concordância entre elas.
		
		\textbf{Palavras-chave:} Cosmologia. Cálculo Numérico.
		 Estatística. Correlações Cruzadas.
	\end{resumo}
	

	
	\begin{SingleSpace}
		\noindent Molina, Arthur de Souza \textbf{Tools for Exploring Cosmic Background Radiation and its Anomalies}. 2021.
		Scientific Initiation Report in Physics \textendash{} Universidade
		Estadual de Londrina, Londrina, 2021.
	\end{SingleSpace}
	
	\setlength{\absparsep}{18pt}
	
	\begin{resumo}[Abstract]
		
		\begin{otherlanguage*}{english}
			With the emergence of several cosmological models, modifications to the standard model to explain phenomena or new physical theories, and the evolution of technology, both in terms of accessibility and processing capacity in computers, there is a demand for cosmology codes that allow for simulations, the predicting results, verifying statistical coherence to observational data, more optimized calculations for datasets that grow exponentially with the production of experiments and the consensus of results obtained in numerical calculations among research groups. In this work, we study the description of the formation of large-scale structures in the universe with a Newtonian treatment through gravitational instability and analyze the codes related to cross-correlations from the CCL and NumCosmo numerical calculus libraries, to produce a \textit{jupyter notebook} using a CCL validation algorithm as a basis to create a test to compare the numerical calculations made by the libraries using the same dataset, document the accuracy between the libraries and the methods with graphs and verify the agreement between them. 
			
			\textbf{Keywords:} Cosmology. Numerical Calculation. Statistics. Cross-correlation.
		\end{otherlanguage*}
		
	\end{resumo}
	
	\pdfbookmark[0]{\listfigurename}{lof}
	
	\listoffigures*
	
	
	%\pdfbookmark[0]{\listtablename}{lot}
	
	%\listoftables*
	
	
	\cleardoublepage
	
	\pdfbookmark[0]{\contentsname}{toc}

	\tableofcontents*
	
	\cleardoublepage
	
	\textual
	\pagenumbering{arabic}
	\setcounter{page}{1}
	
		\chapter*{Introdução}
\addcontentsline{toc}{chapter}{Introdução}

 Ao apontar o telescópio para uma fração do céu e estudar um observável, obtendo o fluxo, o espectro dos feixes de luz provenientes dessa região em diversos pontos no céu, realizando a medida dos dados brutos e após separar o sinal e minimizar os ruídos através de um tratamento estatístico, é possível construir um mapa desse observável e realizar a medida da função de dois pontos.
 
A medida da função de dois pontos apresenta diversos obstáculos e por isso não é tão fácil de ser realizada, pois a medida incorpora alguns efeitos observacionais, como a resolução do instrumento, que precisam ser filtrados para que possam ser comparados com os resultados das previsões teóricas e experimentos diferentes. 

A função de dois pontos, ou seja, a função de correlação permite inferir a distribuição de matéria no Universo, apresentando informações muito importantes para ajustar parâmetros cosmológicos, estudar detalhes da composição, forma e a evolução de estruturas de grandes escalas.

Ao observarmos o universo utilizando escalas próximas ou superiores a 200 Mpc, o universo pode ser aproximado como sendo homogêneo e isotrópico, como descrito pelo princípio cosmológico, mas em escalas menores, o universo possui flutuações de densidade de matéria, como vazios e superaglomerados. A descrição da formação de observáveis relativamente pequenos nessas regiões como planetas, estrelas, galáxias, envolvem uma física bastante complicada.

Neste trabalho será discutido a descrição de estruturas maiores que galáxias, as observáveis que estão apenas em processo de colapso sobre a sua própria gravidade, e devido a este fato, se expandem menos rapidamente em relação à outras regiões, supondo que essas estruturas não possuem velocidades relativísticas, elas podem ser descritas pela teoria Newtoniana com aproximações.

Também será analisado os códigos de correlações cruzadas de ambas as bibliotecas, com  o objetivo de calcular os observáveis principais e suas componentes em um \textit{jupyter notebook}, plotando suas curvas e diferenças relativas a fim de demonstrar a concordância ou discordância entre os resultados das duas bibliotecas.





		
		\chapter*{Instabilidade Gravitacional}
\addcontentsline{toc}{chapter}{Instabilidade Gravitacional}

Quando estudamos uma estrutura de grande escala utilizando as teorias Newtonianas, pressupomos que a estrutura não possui velocidade relativística, a estrutura está em uma região que se expande menos rapidamente em relação as outras regiões, a estrutura esteja em colapso sobre a sua auto-gravidade, ou seja, a instabilidade gravitacional é o mecanismo de maior contribuição para sua evolução e a matéria se comporta como um fluído perfeito.

Os resultados obtidos pela mecânica clássica possuem uma discrepância com um fator de $\frac{v^2}{c^2}$ em relação a mecânica relativística, ao pressupor que a estrutura não possui velocidade relativística, então $\frac{v^2}{c^2} \rightarrow 0$ e a mecânica relativística deve concordar com os resultados obtidos pela mecânica clássica, já o fato que o espaço se expande de forma acelerada em todas as direções como descrito pela lei de Hubble-Lemaître, não impede de fazer uma descrição mais simples do comportamento da matéria e acrescentar conceitos gradualmente com as pertubações na região estudada.

A partir dessas pressuposições, podemos descrever essas estruturas em termos de sua distribuição de energia $\varepsilon(\mathbf{x},t)$, a entropia por unidade de massa $\textbf{S}(\mathbf{x},t)$ e o vetor velocidade $\textbf{V}(\mathbf{x},t)$, ao estabelecer essas quantidades para um volume fixo em uma região, sabemos que a variação da massa deve ser equivalente a variação de sua distribuição de energia em todo o volume nesta região, em outras palavras, 

\begin{equation}\label{eq1}
	\frac{dM}{dt} = \int_{\Delta V} \frac{\partial \varepsilon(\mathbf{x},t)}{\partial t} dV.
\end{equation}

A variação da massa também pode ser escrita como a variação do fluxo da distribuição de energia em todo o contorno do volume fixo

\begin{equation}\label{eq2}
	\frac{dM}{dt} = - \oint \varepsilon(\mathbf{x},t)\mathbf{V} d\sigma = - \int_{\Delta V} \nabla (\varepsilon(\mathbf{x},t)\mathbf{V}) dV.
\end{equation}

Devido a equivalência de eq. (1) e eq. (2), podemos escrever a seguinte relação consistente

\begin{equation}\label{eq3}
	\frac{\partial \varepsilon}{\partial t} + \nabla (\varepsilon\mathbf{V}) = 0,
\end{equation}

que nos permite lidar com a massa em termos de sua distribuição de energia e volume explicitamente. Podemos descrever instabilidade gravitacional na região, utilizando o conceitos como potencial gravitacional $\phi$, a segunda Lei de Newton e a pressão no fluído, a força gravitacional pode ser escrita como

\begin{equation}\label{eq4}
	\textbf{F}_{gr} = - \Delta M \nabla\phi
\end{equation}

a força devido a pressão no fluído é dada por 

\begin{equation}\label{eq5}
	\textbf{F}_{pr} = - \oint p \cdot d\sigma = - \int_{\Delta V} \nabla p\,\,\, dV
\end{equation}

ao utilizar a segunda Lei de Newton, podemos encontrar a equação de Euler

\begin{equation}\label{eq6}
	\dfrac{\partial \textbf{V}}{\partial t} + (\textbf{V} \cdot \nabla) \textbf{V} + \dfrac{\nabla p}{\varepsilon} + \nabla\phi = 0
\end{equation}
 que nos permite descrever a instabilidade gravitacional em termos do volume, distribuição de energia, a pressão e potencial gravitacional.
 
 A conservação de entropia do sistema não permite a dissipação de energia, e portanto, a entropia para um pequeno elemento de matéria é conservada
 
 \begin{equation}\label{eq7}
 	\dfrac{d S(\textbf{x},t)}{dt} = \dfrac{\partial S}{\partial t} + (\textbf{V} \cdot \nabla) S = 0,
 \end{equation} 
 
 também nos utilizamos da equação de Poisson para determinar o potencial gravitacional
 
 \begin{equation}\label{eq8}
 	\nabla^2\phi = 4\pi G\varepsilon.
 \end{equation}

E através dessas equações hidrodinâmicas, podemos estudar o comportamento de pequenas pertubações nas funções desconhecidas: $ \varepsilon $,$ \textbf{V} $, $ p $ e $ \phi $.

\section*{Teoria de Jeans}

Assumimos que o universo é estático, homogêneo, isotrópico, não expansível e que a distribuição de energia na região estudada não varia com tempo e permaneça constante $\varepsilon (\mathbf{x},t) = \text{constante}$, mas para que a distribuição de energia permaneça constante, a matéria necessitaria de estar em repouso, o fato da força gravitacional ser proporcional ao gradiente do potencial não permitindo que essa condição seja satisfeita, ou seja, a equação de Poisson não é satisfeita, logo o universo precisar de uma constante de cosmológica apropriada para se manter estático.

Com pequenas pertubações, temos

\begin{equation}\label{eq9}
	\varepsilon (\textbf{x},t) = \varepsilon_0 + \delta\varepsilon (\textbf{x},t),\,\, \textbf{V} (\textbf{x},t) = \textbf{V}_0 +\delta\textbf{V} (\textbf{x},t) 
\end{equation}

$$\phi (\textbf{x},t) = \phi_0 + \delta\phi (\textbf{x},t), S (\textbf{x},t)= S_0 + \delta S (\textbf{x},t)$$

onde cada variação $\delta\varepsilon \ll \varepsilon_0$ e assim por diante. A pressão é dada por

\begin{equation}\label{eq10}
	p (\textbf{x},t) = p( \varepsilon_0 + \delta\varepsilon (\textbf{x},t), S_0 + \delta S (\textbf{x},t) ) = p_0 +\delta p (\textbf{x},t) 
\end{equation}

ao realizar uma aproximação linear dessas pertubações, podemos escrever a pertubação na pressão em termos da pertubação da distribuição de energia e a entropia assim

\begin{equation}\label{eq11}
	\delta p = c_s^2\delta\varepsilon + \sigma\delta S
\end{equation}

onde $c^2_s \equiv \left(\dfrac{\partial p}{\partial\varepsilon}\right)_s$ é o quadrado da velocidade do som e $\sigma \equiv \left(\dfrac{\partial p}{\partial S}\right)_\varepsilon$. Para a matéria não relativística, a velocidade do som é muito menor que a velocidade da luz ($c_s \ll c  $).

Através dessas pertubações nas variáveis do sistema, conseguimos construir uma descrição clara com aproximações lineares o comportamento do sistema. Ao aplicar as pequenas pertubações em cada uma das definições acima, combinando as equações, conseguimos obter uma relação de cada uma das variações ao mantendo apenas os termos lineares as pertubações da seguinte forma 

\begin{equation}\label{eq12}
	\dfrac{\partial^2\delta\varepsilon}{\partial t^2} - c^2_s\nabla^2\delta\varepsilon - 4\pi G\varepsilon_0\delta\varepsilon = \sigma\nabla^2\delta S(\textbf{x}),
\end{equation}

$ \delta\varepsilon $ está contida em uma equação fechada, onde a entropia serve como uma fonte da pertubação.

No caso de uma pertubação adiabática, onde não há troca de calor, ou seja, a pertubação na entropia é nula, e portanto, não depende das coordenadas espaciais, uma vez que a entropia é o único termo da eq.(12) depende exclusivamente das coordenadas espaciais, consequentemente, o termo a direita da equação eq.(12) se torna equivalente a 0, logo, podemos utilizar as transformações de Fourier e encontrar uma equação diferencial que relaciona a pertubação da distribuição de energia com as quantidades expressas na equação completamente dependente do tempo dada por 

\begin{equation}\label{eq13}
	\dfrac{\partial^2 (\delta\varepsilon_k )}{\partial t^2} +(k^2c^2_s - 4\pi G\varepsilon_0)\delta\varepsilon_k (t) = 0
\end{equation}

a solução é dada por 

\begin{equation}\label{eq14}
	\delta\varepsilon_k (t) \propto e^{\pm \omega (t) i}
\end{equation}

onde $\omega (t) = \sqrt{k^2c^2_s - 4\pi G \varepsilon_0}$ e o comportamento da pertubação adiabática depende exclusivamente do sinal do expoente.

Definindo o comprimento Jeans como

\begin{equation}\label{eq15}
	\lambda_J = \dfrac{2\pi}{k_J} = c_S \left(\dfrac{\pi}{G\varepsilon_0} \right)^{1/2},
\end{equation}

onde $\omega (k_J) = 0$, para  $\lambda < \lambda_J$, as soluções descrevem as ondas sonoras, proporcionais a $ \delta\varepsilon_k \propto \sin (\omega t + \mathbf{k}\mathbf{x} + \alpha) $ que propaga com velocidade de fase 

\begin{equation}\label{eq16}
	c_{fase} = \dfrac{\omega}{k}= c_s\sqrt{1 - \dfrac{k^2_J}{k}}.
\end{equation}

Ao analisar as soluções, no limite de $k \geq k_J$ ou em escalas muito pequenas ($\lambda \leq \lambda_J$), onde a gravidade é insignificante comparado com a pressão, consequentemente $c_{fase} \to c_s$.\\

Uma dessas soluções descrevem o comportamento exponencialmente rápido e não homogêneo, enquanto outras correspondem o modo decaimento, onde $k \to 0$, $|\omega | t \to \dfrac{t}{t_{gr}}$, onde $t_{gr} \equiv (4\pi G\varepsilon_0)^{-1/2}$. Onde $t_{gr}$ é interpretado como o tempo característico de colapso para uma região com uma densidade de energia inicial $\varepsilon_0$.

O comprimento Jeans $\lambda_J \sim c_s t_{gr} $ é o "comunicação de som" sobre a qual a pressão consegue reagir as mudanças da densidade de energia devido ao colapso gravitacional.

No caso de uma pertubação de vetor, onde as pertubações na distribuição de energia e na entropia são nulas, ou seja, as relações lineares de pertubação com $ \delta\varepsilon = 0 $ e $ \delta S = 0$, o resultado dessas substituições, e inclusive no espaço de Fourier, concluímos que a velocidade da pertubação de onda plana, isto é, $\delta\mathbf{v} = \mathbf{w_k} e^{i\mathbf{k}\mathbf{x}}$ e que a velocidade é perpendicular ao vetor de onda $\mathbf{k}$, ou seja,

\begin{equation}\label{eq17}
	\mathbf{w_k} \cdot \mathbf{k} = 0.
\end{equation}

 As perturbações vetoriais descrevem o movimento de cisalhamento do meio que não perturbam a densidade de energia, pois existem duas direções perpendiculares independentes para $\mathbf{k}$.
 
 Por fim, as pertubações de entropia, onde $ \delta S \neq 0$, ao analisar a eq. (12) completa, aplicando as transformações de Fourier obtemos a seguinte equação diferencial 
 
\begin{equation}\label{eq18}
 	\dfrac{\partial^2 (\delta\varepsilon_k )}{\partial t^2} +(k^2c^2_s - 4\pi G\varepsilon_0)\delta\varepsilon_k  = -\sigma k^2 \delta S_k
\end{equation} 
 
 onde a solução geral é a combinação da solução da homogênea e da solução particular, para o caso independente do tempo $  \dfrac{\partial^2 (\delta\varepsilon_k )}{\partial t^2}  = 0 $, a solução é dada por 
 
\begin{equation}\label{eq19}
	\delta\varepsilon_k  = \dfrac{ -\sigma k^2 \delta S_k}{k^2c^2_s - 4\pi G\varepsilon_0}
\end{equation}

é chamada de pertubação de entropia. Nota-se que $k \to \infty$ quando a gravidade não é relevante. Neste caso, a contribuição para a pressão devido à falta de homogeneidade da densidade de energia é exatamente compensada pela contribuição correspondente as perturbações de entropia, logo, essas ocorrem apenas em fluidos constituídos de muitas componentes, por exemplo, um fluido que consiste em bárions e radiação.

Ao considerarmos a instabilidade gravitacional em um universo expansão, homogêneo e isotrópico, onde a distribuição de energia depende do tempo e as velocidades obedecem a lei de Hubble-Lemaître

\begin{equation}\label{eq20}
	\varepsilon = \varepsilon_0 (t), \,\,\, \mathbf{V} = \mathbf{V}_0 = \mathbf{H} (t) \mathbf{x}.
\end{equation}
 
 substituimos esse resultado na eq. (3), obtemos
 
\begin{equation}\label{eq21}
	\dfrac{\partial \varepsilon_0}{\partial t}  + 3 \mathbf{H}\varepsilon_0 = 0 
\end{equation} 

ao agruparmos a divergência equação de Euler com a equação de Poisson, chegamos na equação de Friedmann

\begin{equation}\label{eq22}
	\dot{\mathbf{H}} + \mathbf{H}^2 = - \dfrac{4\pi G}{3}\varepsilon_0.
\end{equation}

As pertubações são dadas por

\begin{equation}\label{eq23}
	\varepsilon = \varepsilon_0 + \delta\varepsilon_0 (\mathbf{x},t),\,\, \mathbf{V} = \mathbf{V}_0 + \delta\mathbf{v} , \phi= \phi_0 + \delta\phi\,\,\, 
\end{equation}

$$p = p_0 + \delta p= p_0 + c_2^2\delta\varepsilon$$

ao ignorar a pertubação de entropia. Podemos relacionar novamente as equações anteriores de hidrodinâmica com os termos lineares as pertubações encontrando relações semelhantes. 

A velocidade de Hubble-Lemaître depende exclusivamente da posição $ \mathbf{x} $, portanto as transformações de Fourier não desacoplam as coordenadas eulerianas $ \mathbf{x} $ em equações diferenciais ordinárias dependentes exclusivamente do tempo. Logo, é mais conveniente utilizar as coordenadas Lagrangianas (comoventes com o fluxo de Hubble) $ \mathbf{q} $ com  

\begin{equation}\label{eq24}
	\mathbf{x} = a(t)\mathbf{q},
\end{equation}

relacionando as operações diferenciais para uma função genérica 

\begin{equation}\label{eq25}
	\left(\dfrac{\partial}{\partial t}\right)_\mathbf{x} = \left( \dfrac{\partial}{\partial t} \right)_\mathbf{q} - (\mathbf{V}_0 \cdot \nabla_\mathbf{x}),
\end{equation}

as devidas espaciais estão relacionadas de forma mais simples

\begin{equation}\label{eq26}
	\nabla_\mathbf{x} = \dfrac{1}{a}\nabla_\mathbf{q}
\end{equation}

e introduzindo a amplitude fracionária de densidade de energia $ \delta \equiv \dfrac{\delta\varepsilon}{\varepsilon_0} $








		\chapter*{Análise dos códigos}
\addcontentsline{toc}{chapter}{Análise dos códigos}

As análises a seguir foram realizadas em um ambiente com uma distribuição Linux baseado em Debian, o Ubuntu e pode ser realizada em qualquer ambiente Linux sem dificuldades com algumas adaptações. A instalação do CCL é simples, consulte as instruções da  \textit{\href{https://ccl.readthedocs.io/en/latest/source/installation.html}{\color{blue}instalação}} do CCL, mas antes instale as dependências, os pacotes \textit{SWIG} e \textit{cmake}, utilizando o gerenciador de pacotes do python. Já a instalação da biblioteca da NumCosmo, consulte as instruções da \textit{\href{https://numcosmo.github.io/download/}{\color{blue}instalação}} da NumCosmo, a orientação de instalação depende da distribuição Linux utilizada.

\begin{comment}		%Apresentação das bibliotecas, citar as passagem dos sites.
\section*{CCL - Core Cosmology Library}
O \href{https://ccl.readthedocs.io/en/latest/?badge=latest#core-cosmology-library}{\color{blue}CCL} ( Core Cosmology Library) é uma biblioteca padronizada de cosmologia que fornece rotinas para computar observáveis cosmológicos básico com alta precisão e foi verificada com um amplo conjunto de testes de validação (\textit{brenchmarks}). As previsões são fornecidas para muitas grandezas cosmológicas, incluindo distâncias, espectro de potência angular, funções de correlação e entre outras  \href{https://arxiv.org/abs/1812.05995}{\color{blue}suportadas}. 

O CCL é escrita em C e Python, com os códigos de cálculo numérico escrito em C e a orientação a objeto escrita em Python, possuindo uma API pública em python sem a necessidade da alteração na interface em C, em um pacote python, o \textit{pyccl}, com módulos intuitivos que permitem computar diversas grandezas cosmológicas suportadas, consulte a   \href{https://ccl.readthedocs.io/en/latest/api/modules.html}{\color{blue}documentação} do CCL.

\section*{NumCosmo}

NumCosmo é uma biblioteca C de software livre cujo objetivo principal é testar modelos cosmológicos usando dados observacionais e fornecer um conjunto de ferramentas para realizar cálculos cosmológicos. A \href{https://numcosmo.github.io/about/}{\color{blue}NumCosmo}  é escrita em C, possui orientação a objeto através do framework \textit{GObject} e\cite{virtualizacao2014} compatibilidade para linguagens que suportam introspecção Gobject, como Python, Perl e entre outros. 
\end{comment}

\section*{O algoritmo de validação}

Para realizar a comparação entre as observáveis, neste trabalho foi utilizado um dos algoritmos do conjunto de teste de validação do CCL, o algoritmo de verificação dos cálculos de correlações cruzadas escrito em python chamado  \textit{test\_correlation.py}.

Este algoritmo consulta um conjunto de dados de redshift, o espectro de potência angular, contraste de densidade de matéria, multipolos correspondentes ao espectro de potência de entrada, spin e entre outras grandezas cosmológicas para inicializar os traçadores, ou seja, as funções de correlação cruzada, calculando o valor da função de correlação para as separações angulares fornecidas como entrada e verifica a coerência do resultado comparando com o valor do erro de cálculo estimulado.

O algoritmo utiliza a estrutura de validação com o módulo \textit{pytest} para realizar 112 testes, utilizando três traçadores:  \textit{NumberCountsTracer}, \textit{WeakLensingTracer} e \textit{CMBLensingTracer}. Os testes foram feitos com os métodos: \textit{fftlog} (Transformações rápidas de Fourier que permite menos custo computacional de processamento do que computar integrações de força bruta nos cálculos), \textit{bessel} (método de cálculo utilizando as funções esféricas de Bessel), já o método \textit{legendre} (Soma da força bruta sobre os polinômios de Legendre) não foi implementado.

O algoritmo consulta 35 arquivos com extensão \textit{.txt} contendo um conjunto de dados, mas apenas 4 arquivos são utilizados para realizar os cálculos dos observáveis e 31 arquivos são consultados para calcular o erro estimado e validar o calculo realizado. O algoritmo possui um problema relativamente simples de implementação relacionado a declaração do endereço do diretório dos arquivos, onde é necessário uma alteração no código para que o teste encontre os arquivos para a consulta e consiga ser executado sem problemas. 

Um dos interesses deste trabalho foi utilizar os testes com traçador\textit{NumberCountsTracer}, ou com apenas mapas com spin-0 utilizando outros traçadores, afim de comparar os resultados do teste com os resultados computados pela NumCosmo utilizando os mesmos conjuntos de dados e verificar a grau de concordância entre as bibliotecas sobre tais resultados, mas infelizmente, o teste com essas condições não foi implementado. A NumCosmo ainda não possui suporte para mapas com spin diferente de zero e, portanto, dentre os testes do algoritmo de validação, apenas os testes com o traçador  \textit{NumberCountsTracer} e mapa com spin-0 nos possibilitam efetuar a análise. O algoritmo possui suporte para computar tanto os casos analíticos quanto os casos de histograma, como os cálculos de casos analíticos não foram implementados na NumCosmo, iremos trabalhar apenas com o casos dados de histograma.

Ao adaptar o teste que utiliza o traçador \textit{NumberCountsTracer} e mapa com spin-0 em um \textit{jupyter notebook}, descartarmos a dependência da consulta dos 31 arquivos para a validação dos resultados, uma vez que os mesmos foram verificados pelo algoritmo original e também foi descartado a inicialização de outros traçadores e seus respectivos cálculos, pois computavam os observáveis de spin diferente de zero.
 
Após as simplificações descritas a cima, surgem dois desafios sobre a estrutura do código para a implementação, que seria remover a dependência do módulo \textit{pytest} e da consulta do conjunto de dados. A estrutura de testes escalonáveis do módulo \textit{pytest} se torna desnecessária, pois não precisamos de realizar uma nova validação dos dados. A consulta do conjunto de dados em arquivos específicos não é interessante, como o conjunto de dados é relativamente pequeno e não há necessidade de transportar arquivos em conjunto com o \textit{jupyter notebook}.

Com auxílio de algoritmos simples escritos em python, utilizando alguns conceitos como \textit{List Comprehension}, abstraímos os dados dos arquivos consultados e produzir listas dentro do jupyter notebook para a consulta, e com um estudo detalhado do comportamento do algoritmo e de suas diversas funções escalonadas para realizar os testes, e várias tentativas de sintetizar o código, simplificamos todas as funções do algoritmo em apenas uma função, com poucas entradas e possibilidade de implementar dos submódulos da NumCosmo com facilidade.







 














		\chapter*{Resultados}
\addcontentsline{toc}{chapter}{Resultados}



		\chapter*{Conclusão}
\addcontentsline{toc}{chapter}{Conclusão}

Através desse trabalho foi possível mostrar a concordância entre os resultados calculados pela NumCosmo e CCL usando o mesmo conjunto de dados, comparando os métodos e funções de correlação com o traçador \textit{NumberCountTracer}. A  curva traçada pelos pontos calculados para cada separação angular fornecida como entrada em gráficos apresentam a distância relativa mostram que os resultados convergem com uma precisão média de $ 10^{-4} $.

Neste trabalho não foi feito apenas o estudo de conceitos relacionados a programação como desenvolver atividades para aprender com mais profundidade as aplicações de \textit{python}, \textit{C/C++}, a orientação à objeto escrita em \textit{Gobject} e as formas diferentes de traduzir \textit{python} para \textit{C/C++}. Também desenvolvemos o estudo dos fundamentos da cosmologia e posteriormente o estudo de estruturas de grande escala no universo e abrir as contas para extrair uma compreensão mais aprofundada usando o livro do Mukhanov \cite{mukhanov}.	
	
	\postextual
	\bibliographystyle{abntex2-num}
	\bibliography{ref}
	
\end{document}
