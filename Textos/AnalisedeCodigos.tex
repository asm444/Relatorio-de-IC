	\chapter*{Análise dos códigos}
\addcontentsline{toc}{chapter}{Análise dos códigos}

As análises a seguir foram realizadas em um ambiente com uma distribuição Linux baseado em Debian, o Ubuntu e pode ser realizada em qualquer ambiente Linux sem dificuldades com algumas adaptações. A instalação do CCL é simples, consulte as instruções da  \textit{\href{https://ccl.readthedocs.io/en/latest/source/installation.html}{\color{blue}instalação}} do CCL, mas antes instale as dependências, os pacotes \textit{SWIG} e \textit{cmake}, utilizando o gerenciador de pacotes do python. Já a instalação da biblioteca da NumCosmo, consulte as instruções da \textit{\href{https://numcosmo.github.io/download/}{\color{blue}instalação}} da NumCosmo, a orientação de instalação depende da distribuição Linux utilizada.

\section*{CCL - Core Cosmology Library}
O \href{https://ccl.readthedocs.io/en/latest/?badge=latest#core-cosmology-library}{\color{blue}CCL} ( Core Cosmology Library) é uma biblioteca padronizada de cosmologia que fornece rotinas para computar observáveis cosmológicos básico com alta precisão e foi verificada com um amplo conjunto de testes de validação (\textit{brenchmarks}). As previsões são fornecidas para muitas grandezas cosmológicas, incluindo distâncias, espectro de potência angular, funções de correlação e entre outras  \href{https://arxiv.org/abs/1812.05995}{\color{blue}suportadas}. 

O CCL é escrita em C e Python, com os códigos de cálculo numérico escrito em C e a orientação a objeto escrita em Python, possuindo uma API pública em python sem a necessidade da alteração na interface em C, em um pacote python, o \textit{pyccl}, com módulos intuitivos que permitem computar diversas grandezas cosmológicas suportadas, consulte a   \href{https://ccl.readthedocs.io/en/latest/api/modules.html}{\color{blue}documentação} do CCL.

\section*{NumCosmo}

A \href{https://numcosmo.github.io/about/}{\color{blue}NumCosmo}  é escrita em C, possui orientação a objeto através do framework \textit{GObject} e compatibilidade para linguagens que suportam introspecção Gobject, como Python, Perl e entre outros. 

\section*{O algoritmo de validação}

Para realizar a comparação entre as observáveis, neste trabalho foi utilizado um dos algoritmos do conjunto de teste de validação do CCL, o algoritmo de verificação dos cálculos de correlações cruzadas escrito em python chamado  \textit{test\_correlation.py}.

Este algoritmo consulta um conjunto de dados de redshift, o espectro de potência angular, contraste de densidade de matéria, multipolos correspondentes ao espectro de potência de entrada, spin e entre outras grandezas cosmológicas para inicializar os traçadores, ou seja, as funções de dois pontos, calculando o valor da função de correlação para as separações angulares fornecidas como entrada e verifica a coerência do resultado comparando  com valor do erro de cálculo estimulado.

O algoritmo utiliza a estrutura de validação com módulo \textit{pytest} para realizar 112 testes, utilizando três traçadores:  \textit{NumberCountsTracer}, \textit{WeakLensingTracer} e \textit{CMBLensingTracer}. Os testes foram feitos com os métodos: \textit{fftlog} (Transformações rápidas de Fourier que permite menos custo computacional de processamento do que computar integrações em força bruta), \textit{bessel} (método de cálculo utilizando as funções esféricas de Bessel), já o método \textit{legendre} (Soma em força bruta sobre os polinômios de Legendre) não foi implementado.

Um dos interesses deste trabalho foi utilizar os cálculos com o traçador \textit{CMBLensingTracer} e spin-0 afim de comparar os resultados utilizando a NumCosmo, infelizmente, o teste com essas condições não foi implementado, O teste consulta quatro.













