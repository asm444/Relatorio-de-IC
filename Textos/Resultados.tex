	\chapter*{Resultados}
\addcontentsline{toc}{chapter}{Resultados}

Escolhemos uma forma padronizada de fácil visualização para a apresentação dos resultados em dois gráficos horizontais, onde o primeiro gráfico apresenta a curva dos valores calculados para as separações angulares fornecidas como entrada e o segundo apresenta a curva da distância relativa entre os resultados calculados.  

\section*{Comparação entre os resultados do CCL}

 Na primeira análise apresentamos os resultados da função de correlação calculada usando os mesmos traçadores mas utilizando métodos diferentes.

\begin{figure}[h]
	\centering
	\includegraphics[width=0.7\linewidth]{"../Produção do Relatório/Trabalho de ic de Arthur/figuras/fig1"}
	\caption{A primeira análise mostra os resultados da comparação entre os métodos \textit{fftlog} e o \textit{bessel} do CCL utilizando os dados de histograma e os mesmos traçadores com o objetivo de apresentar a precisão relativa entre eles.}
	\label{fig:fig1}
\end{figure}

A distância relativa dos valores calculados utilizando métodos diferentes apresenta um grau concordância relativamente bom  e com precisão média de $ 10^{-3} $ entre os valores calculados. 
\newpage
Em seguida ao utilizarmos as curvas dos cálculos utilizando os mesmos métodos e com traçadores diferentes obtemos o seguinte gráfico:

\begin{figure}[h]
	\centering
	\includegraphics[width=0.7\linewidth]{"../Produção do Relatório/Trabalho de ic de Arthur/figuras/fig2"}
	\caption{A segunda análise apresenta a comparação utilizando os dados de histograma, os mesmos métodos e diferentes funções de correlação.}
	\label{fig:fig2}
\end{figure}

Na segunda análise, a distância relativa possuem concordância porém a precisão é extremamente baixa.

Também foi possível comparar os resultados calculados pelo CCL utilizando os dados do caso analítico em diferentes métodos e funções de correlação, na terceira análise, utilizamos os dados dos casos analíticos e comparamos os resultados esperados utilizando métodos diferentes para a mesma função de correlação.

\begin{figure}[h]
	\centering
	\includegraphics[width=0.7\linewidth]{"../Produção do Relatório/Trabalho de ic de Arthur/figuras/fig3"}
	\caption{A terceira análise mostra os resultados da comparação entre os métodos \textit{fftlog} e o \textit{bessel} do CCL utilizando os dados do caso analíticos utilizando a mesma função de correlação, em cada um dos casos, as funções g1 e g2.}
	\label{fig:fig3}
\end{figure}

A distância relativa apresentada na terceira análise demostra que as funções de correlação calculadas utilizando diferentes métodos fornecidos pelo CCL concordam entre si com uma precisão média de $ 10^{-3} $ entre seus resultados. Por fim, comparamos os valores calculados utilizando os mesmos métodos para diferentes funções de correlação e obtemos os seguintes gráficos:

\begin{figure}[H]
	\centering
	\includegraphics[width=0.7\linewidth]{"../Produção do Relatório/Trabalho de ic de Arthur/figuras/fig4"}
	\caption{A quarta análise mostra a comparação dos resultados computados pelo CCL ao utilizar os dados do caso analítico, os mesmos métodos para calcular funções de correlação diferentes no mesmo gráfico.}
	\label{fig:fig4}
\end{figure}

A distância relativa na quarta análise demostra que os resultados na 






