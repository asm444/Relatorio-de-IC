	\chapter*{Introdução}
\addcontentsline{toc}{chapter}{Introdução}

 Ao apontar o telescópio para uma fração do céu e estudar um observável, observando o fluxo, o espectro dos feixes de luz provenientes dessa região em diversos pontos no céu, realizando a medida dos dados brutos e após separar o sinal e minimizar os ruídos através de um tratamento estatístico, é possível construir um mapa desse observável e realizar a medida da função de dois pontos.
 
 A medida da função de dois pontos apresenta diversos obstáculos e por isso não é tão fácil de ser realizada, pois a medida incorpora alguns efeitos observacionais, como a resolução do instrumento, que precisam ser filtrados para que possam ser comparados com os resultados das previsões teóricas e experimentos diferentes. 

A função de dois pontos permite inferir a distribuição de matéria no Universo, apresentando informações muito importantes para ajustar parâmetros cosmológicos, estudar detalhes da composição, forma e a evolução de estruturas de grandes escalas.

Ao observarmos o universo utilizando escalas próximas ou superiores a 200 Mpc, o universo pode ser aproximado como sendo homogêneo e isotrópico, como descrito pelo princípio cosmológico, mas em escalas menores, o universo possui flutuações de densidade de matéria, como vazios e superaglomerados. A descrição da formação de observáveis relativamente pequenos nessas regiões como planetas, estrelas, galáxias, envolvem uma física bastante complicada.

Neste trabalho será discutido a descrição de estruturas maiores que galáxias, as observáveis que estão apenas em processo de colapso sobre a sua própria gravidade, e devido a este fato, se expandem menos rapidamente em relação à outras regiões, como essas estruturas não possuem velocidades relativísticas, podem ser descritas pela teoria Newtoniana com aproximações.

Também será analisado os códigos de correlações cruzadas de ambas as bibliotecas, com  o objetivo de calcular os observáveis principais e suas componentes em um \textit{jupyter notebook}, plotando suas curvas e diferenças relativas a fim de demonstrar a concordância ou discordância entre os resultados das duas bibliotecas.





		