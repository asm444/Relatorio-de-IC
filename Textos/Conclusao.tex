	\chapter*{Conclusão}
\addcontentsline{toc}{chapter}{Conclusão}

Através desse trabalho foi possível mostrar a concordância entre os resultados calculados pela NumCosmo e CCL usando o mesmo conjunto de dados, comparando os métodos e funções de correlação com o traçador \textit{NumberCountTracer}. A  curva traçada pelos pontos calculados para cada separação angular fornecida como entrada em gráficos apresentam a distância relativa mostram que os resultados convergem com uma precisão média de $ 10^{-4} $.

Neste trabalho não foi feito apenas o estudo de conceitos relacionados a programação como desenvolver atividades para aprender com mais profundidade as aplicações de \textit{python}, \textit{C/C++}, a orientação à objeto escrita em \textit{Gobject} e as formas diferentes de traduzir \textit{python} para \textit{C/C++}. Também desenvolvemos o estudo dos fundamentos da cosmologia e posteriormente o estudo de estruturas de grande escala no universo e abrir as contas para extrair uma compreensão mais aprofundada usando o livro do Mukhanov \cite{mukhanov}.